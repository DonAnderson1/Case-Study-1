% Options for packages loaded elsewhere
\PassOptionsToPackage{unicode}{hyperref}
\PassOptionsToPackage{hyphens}{url}
%
\documentclass[
]{article}
\usepackage{amsmath,amssymb}
\usepackage{lmodern}
\usepackage{iftex}
\ifPDFTeX
  \usepackage[T1]{fontenc}
  \usepackage[utf8]{inputenc}
  \usepackage{textcomp} % provide euro and other symbols
\else % if luatex or xetex
  \usepackage{unicode-math}
  \defaultfontfeatures{Scale=MatchLowercase}
  \defaultfontfeatures[\rmfamily]{Ligatures=TeX,Scale=1}
\fi
% Use upquote if available, for straight quotes in verbatim environments
\IfFileExists{upquote.sty}{\usepackage{upquote}}{}
\IfFileExists{microtype.sty}{% use microtype if available
  \usepackage[]{microtype}
  \UseMicrotypeSet[protrusion]{basicmath} % disable protrusion for tt fonts
}{}
\makeatletter
\@ifundefined{KOMAClassName}{% if non-KOMA class
  \IfFileExists{parskip.sty}{%
    \usepackage{parskip}
  }{% else
    \setlength{\parindent}{0pt}
    \setlength{\parskip}{6pt plus 2pt minus 1pt}}
}{% if KOMA class
  \KOMAoptions{parskip=half}}
\makeatother
\usepackage{xcolor}
\IfFileExists{xurl.sty}{\usepackage{xurl}}{} % add URL line breaks if available
\IfFileExists{bookmark.sty}{\usepackage{bookmark}}{\usepackage{hyperref}}
\hypersetup{
  pdftitle={Case Study 1},
  pdfauthor={Donald Anderson},
  hidelinks,
  pdfcreator={LaTeX via pandoc}}
\urlstyle{same} % disable monospaced font for URLs
\usepackage[margin=1in]{geometry}
\usepackage{color}
\usepackage{fancyvrb}
\newcommand{\VerbBar}{|}
\newcommand{\VERB}{\Verb[commandchars=\\\{\}]}
\DefineVerbatimEnvironment{Highlighting}{Verbatim}{commandchars=\\\{\}}
% Add ',fontsize=\small' for more characters per line
\usepackage{framed}
\definecolor{shadecolor}{RGB}{248,248,248}
\newenvironment{Shaded}{\begin{snugshade}}{\end{snugshade}}
\newcommand{\AlertTok}[1]{\textcolor[rgb]{0.94,0.16,0.16}{#1}}
\newcommand{\AnnotationTok}[1]{\textcolor[rgb]{0.56,0.35,0.01}{\textbf{\textit{#1}}}}
\newcommand{\AttributeTok}[1]{\textcolor[rgb]{0.77,0.63,0.00}{#1}}
\newcommand{\BaseNTok}[1]{\textcolor[rgb]{0.00,0.00,0.81}{#1}}
\newcommand{\BuiltInTok}[1]{#1}
\newcommand{\CharTok}[1]{\textcolor[rgb]{0.31,0.60,0.02}{#1}}
\newcommand{\CommentTok}[1]{\textcolor[rgb]{0.56,0.35,0.01}{\textit{#1}}}
\newcommand{\CommentVarTok}[1]{\textcolor[rgb]{0.56,0.35,0.01}{\textbf{\textit{#1}}}}
\newcommand{\ConstantTok}[1]{\textcolor[rgb]{0.00,0.00,0.00}{#1}}
\newcommand{\ControlFlowTok}[1]{\textcolor[rgb]{0.13,0.29,0.53}{\textbf{#1}}}
\newcommand{\DataTypeTok}[1]{\textcolor[rgb]{0.13,0.29,0.53}{#1}}
\newcommand{\DecValTok}[1]{\textcolor[rgb]{0.00,0.00,0.81}{#1}}
\newcommand{\DocumentationTok}[1]{\textcolor[rgb]{0.56,0.35,0.01}{\textbf{\textit{#1}}}}
\newcommand{\ErrorTok}[1]{\textcolor[rgb]{0.64,0.00,0.00}{\textbf{#1}}}
\newcommand{\ExtensionTok}[1]{#1}
\newcommand{\FloatTok}[1]{\textcolor[rgb]{0.00,0.00,0.81}{#1}}
\newcommand{\FunctionTok}[1]{\textcolor[rgb]{0.00,0.00,0.00}{#1}}
\newcommand{\ImportTok}[1]{#1}
\newcommand{\InformationTok}[1]{\textcolor[rgb]{0.56,0.35,0.01}{\textbf{\textit{#1}}}}
\newcommand{\KeywordTok}[1]{\textcolor[rgb]{0.13,0.29,0.53}{\textbf{#1}}}
\newcommand{\NormalTok}[1]{#1}
\newcommand{\OperatorTok}[1]{\textcolor[rgb]{0.81,0.36,0.00}{\textbf{#1}}}
\newcommand{\OtherTok}[1]{\textcolor[rgb]{0.56,0.35,0.01}{#1}}
\newcommand{\PreprocessorTok}[1]{\textcolor[rgb]{0.56,0.35,0.01}{\textit{#1}}}
\newcommand{\RegionMarkerTok}[1]{#1}
\newcommand{\SpecialCharTok}[1]{\textcolor[rgb]{0.00,0.00,0.00}{#1}}
\newcommand{\SpecialStringTok}[1]{\textcolor[rgb]{0.31,0.60,0.02}{#1}}
\newcommand{\StringTok}[1]{\textcolor[rgb]{0.31,0.60,0.02}{#1}}
\newcommand{\VariableTok}[1]{\textcolor[rgb]{0.00,0.00,0.00}{#1}}
\newcommand{\VerbatimStringTok}[1]{\textcolor[rgb]{0.31,0.60,0.02}{#1}}
\newcommand{\WarningTok}[1]{\textcolor[rgb]{0.56,0.35,0.01}{\textbf{\textit{#1}}}}
\usepackage{graphicx}
\makeatletter
\def\maxwidth{\ifdim\Gin@nat@width>\linewidth\linewidth\else\Gin@nat@width\fi}
\def\maxheight{\ifdim\Gin@nat@height>\textheight\textheight\else\Gin@nat@height\fi}
\makeatother
% Scale images if necessary, so that they will not overflow the page
% margins by default, and it is still possible to overwrite the defaults
% using explicit options in \includegraphics[width, height, ...]{}
\setkeys{Gin}{width=\maxwidth,height=\maxheight,keepaspectratio}
% Set default figure placement to htbp
\makeatletter
\def\fps@figure{htbp}
\makeatother
\setlength{\emergencystretch}{3em} % prevent overfull lines
\providecommand{\tightlist}{%
  \setlength{\itemsep}{0pt}\setlength{\parskip}{0pt}}
\setcounter{secnumdepth}{-\maxdimen} % remove section numbering
\usepackage{booktabs}
\usepackage{longtable}
\usepackage{array}
\usepackage{multirow}
\usepackage{wrapfig}
\usepackage{float}
\usepackage{colortbl}
\usepackage{pdflscape}
\usepackage{tabu}
\usepackage{threeparttable}
\usepackage{threeparttablex}
\usepackage[normalem]{ulem}
\usepackage{makecell}
\usepackage{xcolor}
\ifLuaTeX
  \usepackage{selnolig}  % disable illegal ligatures
\fi

\title{Case Study 1}
\author{Donald Anderson}
\date{}

\begin{document}
\maketitle

\hypertarget{introduction}{%
\subsection{Introduction}\label{introduction}}

In this case study, I investigated ``The Beers Data Set''. The data set
contains a list of 2,410 US craft beers and Breweries contain 558 US
breweries. I provide my recommendations to a commercial beer company
regarding the top potential markets for their new beer launch based on 9
key insights derived from the data.

\hypertarget{import-data}{%
\subsection{Import Data}\label{import-data}}

\begin{Shaded}
\begin{Highlighting}[]
\NormalTok{beers}\OtherTok{\textless{}{-}} \FunctionTok{read.csv}\NormalTok{(}\AttributeTok{file =} \StringTok{"Beers.csv"}\NormalTok{)}
\NormalTok{Breweries}\OtherTok{\textless{}{-}} \FunctionTok{read.csv}\NormalTok{(}\AttributeTok{file =} \StringTok{"Breweries.csv"}\NormalTok{)}
\end{Highlighting}
\end{Shaded}

\begin{enumerate}
\def\labelenumi{\arabic{enumi}.}
\tightlist
\item
  How many breweries are present in each state?
\end{enumerate}

\begin{Shaded}
\begin{Highlighting}[]
\NormalTok{Breweries}\SpecialCharTok{\%\textgreater{}\%}
  \FunctionTok{group\_by}\NormalTok{(State)}\SpecialCharTok{\%\textgreater{}\%}
  \FunctionTok{count}\NormalTok{(}\AttributeTok{name =} \StringTok{"NoBreweries"}\NormalTok{)}\SpecialCharTok{\%\textgreater{}\%}
  \FunctionTok{arrange}\NormalTok{(}\FunctionTok{desc}\NormalTok{(NoBreweries))}\SpecialCharTok{\%\textgreater{}\%}
\NormalTok{  kableExtra}\SpecialCharTok{::}\FunctionTok{kable}\NormalTok{(}\AttributeTok{format =} \StringTok{"html"}\NormalTok{, }\AttributeTok{padding =} \DecValTok{0}\NormalTok{, }\AttributeTok{caption =} \StringTok{"Breweries per State"}\NormalTok{)}
\end{Highlighting}
\end{Shaded}

Breweries per State

State

NoBreweries

CO

47

CA

39

MI

32

OR

29

TX

28

PA

25

MA

23

WA

23

IN

22

WI

20

NC

19

IL

18

NY

16

VA

16

FL

15

OH

15

MN

12

AZ

11

VT

10

ME

9

MO

9

MT

9

CT

8

AK

7

GA

7

MD

7

OK

6

IA

5

ID

5

LA

5

NE

5

RI

5

HI

4

KY

4

NM

4

SC

4

UT

4

WY

4

AL

3

KS

3

NH

3

NJ

3

TN

3

AR

2

DE

2

MS

2

NV

2

DC

1

ND

1

SD

1

WV

1

Colorado has the most number of breweries.

\begin{Shaded}
\begin{Highlighting}[]
\NormalTok{kk }\OtherTok{\textless{}{-}}\NormalTok{ Breweries}\SpecialCharTok{\%\textgreater{}\%}
  \FunctionTok{group\_by}\NormalTok{(State)}\SpecialCharTok{\%\textgreater{}\%}
  \FunctionTok{summarize}\NormalTok{(}\AttributeTok{n=}\FunctionTok{n}\NormalTok{())}
  
\NormalTok{kk }\OtherTok{\textless{}{-}}\NormalTok{ kk}\SpecialCharTok{\%\textgreater{}\%}
\NormalTok{dplyr}\SpecialCharTok{::} \FunctionTok{mutate}\NormalTok{(}\AttributeTok{State =} \FunctionTok{fct\_reorder}\NormalTok{(State,n))}

  \FunctionTok{ggplot}\NormalTok{(}\AttributeTok{data =}\NormalTok{ kk)}\SpecialCharTok{+}
   \FunctionTok{aes}\NormalTok{(}\AttributeTok{y=}\NormalTok{n, }\AttributeTok{x=}\NormalTok{State)}\SpecialCharTok{+}
  \FunctionTok{geom\_bar}\NormalTok{(}\AttributeTok{stat =} \StringTok{"identity"}\NormalTok{)}\SpecialCharTok{+}
  \FunctionTok{labs}\NormalTok{(}\AttributeTok{title =} \StringTok{"Breweries per State"}\NormalTok{,}
       \AttributeTok{x=}\StringTok{"Breweries"}\NormalTok{)}\SpecialCharTok{+}
  \FunctionTok{theme\_bw}\NormalTok{()}\SpecialCharTok{+}
   \FunctionTok{theme}\NormalTok{(}\AttributeTok{axis.text.x =} \FunctionTok{element\_text}\NormalTok{(}\AttributeTok{angle =} \DecValTok{90}\NormalTok{, }\AttributeTok{vjust =} \FloatTok{0.5}\NormalTok{, }\AttributeTok{hjust=}\DecValTok{1}\NormalTok{))}
\end{Highlighting}
\end{Shaded}

\includegraphics{Unit_8_9_Case_Study_files/figure-latex/unnamed-chunk-3-1.pdf}

\begin{enumerate}
\def\labelenumi{\arabic{enumi}.}
\setcounter{enumi}{1}
\tightlist
\item
  Merge beer data with the breweries data. Print the first 6
  observations and the last six observations to check the merged file.
  (RMD only, this does not need to be included in the presentation or
  the deck.)
\end{enumerate}

\begin{Shaded}
\begin{Highlighting}[]
\NormalTok{merged}\OtherTok{\textless{}{-}} \FunctionTok{left\_join}\NormalTok{(beers, Breweries, }\AttributeTok{by=} \FunctionTok{c}\NormalTok{(}\StringTok{"Brewery\_id"}\OtherTok{=} \StringTok{"Brew\_ID"}\NormalTok{))}\SpecialCharTok{\%\textgreater{}\%}
  \FunctionTok{rename}\NormalTok{(}\StringTok{"BeerName"}\OtherTok{=} \StringTok{"Name.x"}\NormalTok{, }\AttributeTok{Brewery.Name=}\StringTok{"Name.y"}\NormalTok{ )}
\FunctionTok{head}\NormalTok{(merged, }\DecValTok{6}\NormalTok{)}\SpecialCharTok{\%\textgreater{}\%}
\NormalTok{  kableExtra}\SpecialCharTok{::}\FunctionTok{kable}\NormalTok{(}\AttributeTok{format =} \StringTok{"html"}\NormalTok{, }\AttributeTok{padding =} \DecValTok{0}\NormalTok{)}
\end{Highlighting}
\end{Shaded}

BeerName

Beer\_ID

ABV

IBU

Brewery\_id

Style

Ounces

Brewery.Name

City

State

Pub Beer

1436

0.050

NA

409

American Pale Lager

12

10 Barrel Brewing Company

Bend

OR

Devil's Cup

2265

0.066

NA

178

American Pale Ale (APA)

12

18th Street Brewery

Gary

IN

Rise of the Phoenix

2264

0.071

NA

178

American IPA

12

18th Street Brewery

Gary

IN

Sinister

2263

0.090

NA

178

American Double / Imperial IPA

12

18th Street Brewery

Gary

IN

Sex and Candy

2262

0.075

NA

178

American IPA

12

18th Street Brewery

Gary

IN

Black Exodus

2261

0.077

NA

178

Oatmeal Stout

12

18th Street Brewery

Gary

IN

\begin{Shaded}
\begin{Highlighting}[]
\FunctionTok{tail}\NormalTok{(merged, }\DecValTok{6}\NormalTok{)}\SpecialCharTok{\%\textgreater{}\%}
\NormalTok{  kableExtra}\SpecialCharTok{::}\FunctionTok{kable}\NormalTok{(}\AttributeTok{format =} \StringTok{"html"}\NormalTok{, }\AttributeTok{padding =} \DecValTok{1}\NormalTok{)}
\end{Highlighting}
\end{Shaded}

BeerName

Beer\_ID

ABV

IBU

Brewery\_id

Style

Ounces

Brewery.Name

City

State

2405

Rocky Mountain Oyster Stout

1035

0.075

NA

425

American Stout

12

Wynkoop Brewing Company

Denver

CO

2406

Belgorado

928

0.067

45

425

Belgian IPA

12

Wynkoop Brewing Company

Denver

CO

2407

Rail Yard Ale

807

0.052

NA

425

American Amber / Red Ale

12

Wynkoop Brewing Company

Denver

CO

2408

B3K Black Lager

620

0.055

NA

425

Schwarzbier

12

Wynkoop Brewing Company

Denver

CO

2409

Silverback Pale Ale

145

0.055

40

425

American Pale Ale (APA)

12

Wynkoop Brewing Company

Denver

CO

2410

Rail Yard Ale (2009)

84

0.052

NA

425

American Amber / Red Ale

12

Wynkoop Brewing Company

Denver

CO

The head and the tail functions show how the first 6 and the last 6 rows
of the data look.

\begin{enumerate}
\def\labelenumi{\arabic{enumi}.}
\setcounter{enumi}{2}
\tightlist
\item
  Address the missing values in each column.
\end{enumerate}

\begin{Shaded}
\begin{Highlighting}[]
\CommentTok{\# Construct linear model based on non{-}NA pairs}
\NormalTok{df2 }\OtherTok{\textless{}{-}}\NormalTok{ merged }\SpecialCharTok{\%\textgreater{}\%} \FunctionTok{filter}\NormalTok{(}\SpecialCharTok{!}\FunctionTok{is.na}\NormalTok{(IBU) , }\SpecialCharTok{!}\FunctionTok{is.na}\NormalTok{(ABV) )}

\NormalTok{fit }\OtherTok{\textless{}{-}} \FunctionTok{lm}\NormalTok{(IBU }\SpecialCharTok{\textasciitilde{}}\NormalTok{ ABV, }\AttributeTok{data =}\NormalTok{ df2)}
\NormalTok{fit2 }\OtherTok{\textless{}{-}} \FunctionTok{lm}\NormalTok{( ABV}\SpecialCharTok{\textasciitilde{}}\NormalTok{IBU, }\AttributeTok{data =}\NormalTok{ df2)}

\FunctionTok{summary}\NormalTok{(fit)}
\end{Highlighting}
\end{Shaded}

\begin{verbatim}
## 
## Call:
## lm(formula = IBU ~ ABV, data = df2)
## 
## Residuals:
##     Min      1Q  Median      3Q     Max 
## -78.849 -11.977  -0.721  13.997  93.458 
## 
## Coefficients:
##             Estimate Std. Error t value Pr(>|t|)    
## (Intercept)  -34.099      2.326  -14.66   <2e-16 ***
## ABV         1282.037     37.860   33.86   <2e-16 ***
## ---
## Signif. codes:  0 '***' 0.001 '**' 0.01 '*' 0.05 '.' 0.1 ' ' 1
## 
## Residual standard error: 19.26 on 1403 degrees of freedom
## Multiple R-squared:  0.4497, Adjusted R-squared:  0.4493 
## F-statistic:  1147 on 1 and 1403 DF,  p-value: < 2.2e-16
\end{verbatim}

\begin{Shaded}
\begin{Highlighting}[]
\CommentTok{\# Use fit to predict the value}
\NormalTok{df3 }\OtherTok{\textless{}{-}}\NormalTok{ merged }\SpecialCharTok{\%\textgreater{}\%} 
  \FunctionTok{mutate}\NormalTok{(}\AttributeTok{pred =} \FunctionTok{predict}\NormalTok{(fit, .),}
         \AttributeTok{pred2=} \FunctionTok{predict}\NormalTok{(fit2, .) ) }\SpecialCharTok{\%\textgreater{}\%}
  \CommentTok{\# Replace NA with pred in var1}
  \FunctionTok{mutate}\NormalTok{(}\AttributeTok{IBU =} \FunctionTok{ifelse}\NormalTok{(}\FunctionTok{is.na}\NormalTok{(IBU), pred, IBU),}
         \AttributeTok{ABV =} \FunctionTok{ifelse}\NormalTok{(}\FunctionTok{is.na}\NormalTok{(ABV), pred, ABV),}
         \AttributeTok{pred=} \ConstantTok{NULL}\NormalTok{,}
         \AttributeTok{pred2=} \ConstantTok{NULL}\NormalTok{)}\SpecialCharTok{\%\textgreater{}\%}
  \FunctionTok{drop\_na}\NormalTok{()}
\end{Highlighting}
\end{Shaded}

To deal with missing values in ABV and IBU variables we used linear
regression to predict the missing data. In cases where the values were
not predictable, that is where both ABV an IBU were missing, the rows
were removed form the data set.

\begin{enumerate}
\def\labelenumi{\arabic{enumi}.}
\setcounter{enumi}{3}
\tightlist
\item
  Compute the median alcohol content and international bitterness unit
  for each state. Plot a bar chart to compare.
\end{enumerate}

\begin{Shaded}
\begin{Highlighting}[]
\NormalTok{mediandf}\OtherTok{\textless{}{-}}\NormalTok{merged}\SpecialCharTok{\%\textgreater{}\%}
  \FunctionTok{group\_by}\NormalTok{(State)}\SpecialCharTok{\%\textgreater{}\%}
  \FunctionTok{summarise}\NormalTok{(}\AttributeTok{IBU=} \FunctionTok{median}\NormalTok{(IBU, }\AttributeTok{na.rm =}\NormalTok{ T), }\AttributeTok{ABV=} \FunctionTok{median}\NormalTok{(ABV, }\AttributeTok{na.rm =}\NormalTok{ T))}\SpecialCharTok{\%\textgreater{}\%}
  \FunctionTok{mutate}\NormalTok{(}\AttributeTok{State =} \FunctionTok{fct\_reorder}\NormalTok{(State, IBU)) }

\FunctionTok{ggplot}\NormalTok{(}\AttributeTok{data =}\NormalTok{ mediandf, }\FunctionTok{aes}\NormalTok{(}\AttributeTok{y=}\NormalTok{IBU, }\AttributeTok{x=}\NormalTok{ State, }\AttributeTok{fill=}\NormalTok{ State))}\SpecialCharTok{+}
  \FunctionTok{geom\_bar}\NormalTok{(}\AttributeTok{stat =} \StringTok{"identity"}\NormalTok{, }\AttributeTok{show.legend =}\NormalTok{ F)}\SpecialCharTok{+}
  \FunctionTok{labs}\NormalTok{(}\AttributeTok{title =} \StringTok{"IBU by State"}\NormalTok{)}\SpecialCharTok{+}
  \FunctionTok{theme\_bw}\NormalTok{()}\SpecialCharTok{+}
  \FunctionTok{theme}\NormalTok{(}\AttributeTok{axis.text.x =} \FunctionTok{element\_text}\NormalTok{(}\AttributeTok{angle =} \DecValTok{90}\NormalTok{, }\AttributeTok{vjust =} \FloatTok{0.5}\NormalTok{, }\AttributeTok{hjust=}\DecValTok{1}\NormalTok{))}
\end{Highlighting}
\end{Shaded}

\begin{verbatim}
## Warning: Removed 1 rows containing missing values (position_stack).
\end{verbatim}

\includegraphics{Unit_8_9_Case_Study_files/figure-latex/unnamed-chunk-6-1.pdf}

\begin{Shaded}
\begin{Highlighting}[]
\FunctionTok{ggplot}\NormalTok{(}\AttributeTok{data =}\NormalTok{ mediandf, }\FunctionTok{aes}\NormalTok{(}\AttributeTok{y=}\NormalTok{ABV, }\AttributeTok{x=}\NormalTok{ State, }\AttributeTok{fill=}\NormalTok{ State))}\SpecialCharTok{+}
  \FunctionTok{geom\_bar}\NormalTok{(}\AttributeTok{stat =} \StringTok{"identity"}\NormalTok{, }\AttributeTok{show.legend =}\NormalTok{ F)}\SpecialCharTok{+}
  \FunctionTok{labs}\NormalTok{(}\AttributeTok{title =} \StringTok{"ABV by State"}\NormalTok{)}\SpecialCharTok{+}
  \FunctionTok{theme\_bw}\NormalTok{()}\SpecialCharTok{+}
  \FunctionTok{theme}\NormalTok{(}\AttributeTok{axis.text.x =} \FunctionTok{element\_text}\NormalTok{(}\AttributeTok{angle =} \DecValTok{90}\NormalTok{, }\AttributeTok{vjust =} \FloatTok{0.5}\NormalTok{, }\AttributeTok{hjust=}\DecValTok{1}\NormalTok{))}
\end{Highlighting}
\end{Shaded}

\includegraphics{Unit_8_9_Case_Study_files/figure-latex/unnamed-chunk-6-2.pdf}
Note: Removed 1 row containing missing values (position\_stack) from the
IBU bar chart.

\begin{enumerate}
\def\labelenumi{\arabic{enumi}.}
\setcounter{enumi}{4}
\tightlist
\item
  Which state has the maximum alcoholic (ABV) beer? Which state has the
  most bitter (IBU) beer?
\end{enumerate}

\begin{Shaded}
\begin{Highlighting}[]
\NormalTok{maxdf}\OtherTok{\textless{}{-}}\NormalTok{df3}\SpecialCharTok{\%\textgreater{}\%}
  \FunctionTok{group\_by}\NormalTok{(State)}\SpecialCharTok{\%\textgreater{}\%}
  \FunctionTok{summarise}\NormalTok{(}\AttributeTok{IBU=} \FunctionTok{mean}\NormalTok{(IBU), }\AttributeTok{ABV=} \FunctionTok{mean}\NormalTok{(ABV))}

\NormalTok{merged}\SpecialCharTok{$}\NormalTok{State[}\FunctionTok{which.max}\NormalTok{(merged}\SpecialCharTok{$}\NormalTok{ABV)]}
\end{Highlighting}
\end{Shaded}

\begin{verbatim}
## [1] " CO"
\end{verbatim}

\begin{Shaded}
\begin{Highlighting}[]
\NormalTok{merged}\SpecialCharTok{$}\NormalTok{State[}\FunctionTok{which.max}\NormalTok{(merged}\SpecialCharTok{$}\NormalTok{IBU)]}
\end{Highlighting}
\end{Shaded}

\begin{verbatim}
## [1] " OR"
\end{verbatim}

Colorado state has the maximum alcoholic (ABV) beer while Oregon state
has the most bitter beers.

\begin{enumerate}
\def\labelenumi{\arabic{enumi}.}
\setcounter{enumi}{5}
\tightlist
\item
  Comment on the summary statistics and distribution of the ABV
  variable.
\end{enumerate}

\begin{Shaded}
\begin{Highlighting}[]
\NormalTok{psych}\SpecialCharTok{::}\FunctionTok{describe}\NormalTok{(df3}\SpecialCharTok{$}\NormalTok{ABV)}\SpecialCharTok{\%\textgreater{}\%}
\NormalTok{  kableExtra}\SpecialCharTok{::}\FunctionTok{kable}\NormalTok{(}\AttributeTok{format =} \StringTok{"html"}\NormalTok{, }\AttributeTok{padding =} \DecValTok{1}\NormalTok{)}
\end{Highlighting}
\end{Shaded}

vars

n

mean

sd

median

trimmed

mad

min

max

range

skew

kurtosis

se

X1

1

2348

0.0597734

0.0135417

0.056

0.0583888

0.0118608

0.001

0.128

0.127

0.9572529

1.13642

0.0002795

\begin{Shaded}
\begin{Highlighting}[]
\FunctionTok{hist}\NormalTok{(df3}\SpecialCharTok{$}\NormalTok{ABV, }\AttributeTok{breaks =} \DecValTok{10}\NormalTok{, }\AttributeTok{main =} \StringTok{"ABV"}\NormalTok{, }\AttributeTok{xlab =} \StringTok{"ABV"}\NormalTok{, }\AttributeTok{col =} \StringTok{"gray"}\NormalTok{)}
\end{Highlighting}
\end{Shaded}

\includegraphics{Unit_8_9_Case_Study_files/figure-latex/unnamed-chunk-8-1.pdf}

Alcohol by volume of the beer has a mean of 0.06 with a standard
deviation of 0.01. The histogram shows that the data is slightly skewed
to the right, in other words the data is positively skewed.

\begin{enumerate}
\def\labelenumi{\arabic{enumi}.}
\setcounter{enumi}{6}
\tightlist
\item
  Is there an apparent relationship between the bitterness of the beer
  and its alcoholic content? Draw a scatter plot. Make your best
  judgment of a relationship and EXPLAIN your answer.
\end{enumerate}

\begin{Shaded}
\begin{Highlighting}[]
\FunctionTok{ggplot}\NormalTok{(}\AttributeTok{data =}\NormalTok{ df3, }\FunctionTok{aes}\NormalTok{(}\AttributeTok{x=}\NormalTok{ ABV, }\AttributeTok{y=}\NormalTok{IBU))}\SpecialCharTok{+}
  \FunctionTok{geom\_point}\NormalTok{()}\SpecialCharTok{+}
  \FunctionTok{geom\_smooth}\NormalTok{(}\AttributeTok{se=}\NormalTok{F, }\AttributeTok{method =} \StringTok{"lm"}\NormalTok{)}\SpecialCharTok{+}
  \FunctionTok{labs}\NormalTok{(}\AttributeTok{title =} \StringTok{"ABV vs IBU"}\NormalTok{,}
       \AttributeTok{y=}\StringTok{"International Bitterness Units of the beer"}\NormalTok{,}
       \AttributeTok{x=} \StringTok{"Alcohol by volume of the beer"}\NormalTok{)}\SpecialCharTok{+}
  \FunctionTok{theme\_bw}\NormalTok{()}
\end{Highlighting}
\end{Shaded}

\begin{verbatim}
## `geom_smooth()` using formula 'y ~ x'
\end{verbatim}

\includegraphics{Unit_8_9_Case_Study_files/figure-latex/unnamed-chunk-9-1.pdf}

The plot show a strong positive relationship between ABV and IBU in
beers. This mean as ABV increase IBU will also increase.

\begin{enumerate}
\def\labelenumi{\arabic{enumi}.}
\setcounter{enumi}{7}
\tightlist
\item
  Budweiser would also like to investigate the difference with respect
  to IBU and ABV between IPAs (India Pale Ales) and other types of Ale
  (any beer with ``Ale'' in its name other than IPA). You decide to use
  KNN classification to investigate this relationship. Provide
  statistical evidence one way or the other. You can of course assume
  your audience is comfortable with percentages \ldots{} KNN is very
  easy to understand conceptually.
\end{enumerate}

\begin{Shaded}
\begin{Highlighting}[]
\CommentTok{\#}
\NormalTok{KNNDF}\OtherTok{\textless{}{-}}\NormalTok{ df3}\SpecialCharTok{\%\textgreater{}\%}
  \FunctionTok{filter}\NormalTok{(}\FunctionTok{grepl}\NormalTok{(}\StringTok{" Ale "}\NormalTok{, Style, }\AttributeTok{ignore.case =}\NormalTok{ T) }\SpecialCharTok{|} \FunctionTok{grepl}\NormalTok{(}\StringTok{"Ipa"}\NormalTok{, df3}\SpecialCharTok{$}\NormalTok{Style, }\AttributeTok{ignore.case =}\NormalTok{ T))}\SpecialCharTok{\%\textgreater{}\%}
  \FunctionTok{mutate}\NormalTok{(}\AttributeTok{type=} \FunctionTok{case\_when}\NormalTok{( }\FunctionTok{grepl}\NormalTok{(}\StringTok{"Ipa"}\NormalTok{, Style, }\AttributeTok{ignore.case =}\NormalTok{ T)}\SpecialCharTok{\textasciitilde{}}\StringTok{"Ipa"}\NormalTok{,}
                          \ConstantTok{TRUE}\SpecialCharTok{\textasciitilde{}} \StringTok{"Ale"}\NormalTok{))}\SpecialCharTok{\%\textgreater{}\%}
  \FunctionTok{select}\NormalTok{(ABV,IBU,type)}

\FunctionTok{set.seed}\NormalTok{(}\DecValTok{300}\NormalTok{)}
\CommentTok{\#Splitting data as training and test set. Using createDataPartition() function from caret}
\NormalTok{indxTrain }\OtherTok{\textless{}{-}} \FunctionTok{createDataPartition}\NormalTok{(}\AttributeTok{y =}\NormalTok{ KNNDF}\SpecialCharTok{$}\NormalTok{type,}\AttributeTok{p =} \FloatTok{0.75}\NormalTok{,}\AttributeTok{list =} \ConstantTok{FALSE}\NormalTok{)}
\NormalTok{training }\OtherTok{\textless{}{-}}\NormalTok{ KNNDF[indxTrain,]}
\NormalTok{testing }\OtherTok{\textless{}{-}}\NormalTok{ KNNDF[}\SpecialCharTok{{-}}\NormalTok{indxTrain,]}


\FunctionTok{set.seed}\NormalTok{(}\DecValTok{400}\NormalTok{)}
\NormalTok{ctrl }\OtherTok{\textless{}{-}} \FunctionTok{trainControl}\NormalTok{(}\AttributeTok{method=}\StringTok{"repeatedcv"}\NormalTok{,}\AttributeTok{repeats =} \DecValTok{10}\NormalTok{) }
\CommentTok{\#,classProbs=TRUE,summaryFunction = twoClassSummary)}
\NormalTok{knnFit }\OtherTok{\textless{}{-}} \FunctionTok{train}\NormalTok{(type }\SpecialCharTok{\textasciitilde{}}\NormalTok{ ., }\AttributeTok{data =}\NormalTok{ training, }\AttributeTok{method =} \StringTok{"knn"}\NormalTok{, }
                \AttributeTok{trControl =}\NormalTok{ ctrl, }\AttributeTok{preProcess =} \FunctionTok{c}\NormalTok{(}\StringTok{"center"}\NormalTok{,}\StringTok{"scale"}\NormalTok{), }\AttributeTok{tuneLength =} \DecValTok{20}\NormalTok{)}

\CommentTok{\#Output of kNN fit}
\NormalTok{knnFit}
\end{Highlighting}
\end{Shaded}

\begin{verbatim}
## k-Nearest Neighbors 
## 
## 611 samples
##   2 predictor
##   2 classes: 'Ale', 'Ipa' 
## 
## Pre-processing: centered (2), scaled (2) 
## Resampling: Cross-Validated (10 fold, repeated 10 times) 
## Summary of sample sizes: 550, 550, 550, 550, 550, 550, ... 
## Resampling results across tuning parameters:
## 
##   k   Accuracy   Kappa    
##    5  0.8274987  0.6031948
##    7  0.8284691  0.6037750
##    9  0.8235431  0.5890908
##   11  0.8258250  0.5922936
##   13  0.8196087  0.5747189
##   15  0.8241962  0.5872754
##   17  0.8287731  0.5974090
##   19  0.8258329  0.5900467
##   21  0.8241909  0.5826191
##   23  0.8210973  0.5695296
##   25  0.8229059  0.5722713
##   27  0.8232232  0.5738801
##   29  0.8229032  0.5742170
##   31  0.8224167  0.5738847
##   33  0.8220809  0.5740062
##   35  0.8214305  0.5735497
##   37  0.8219143  0.5765143
##   39  0.8237176  0.5823097
##   41  0.8229059  0.5811887
##   43  0.8269937  0.5910321
## 
## Accuracy was used to select the optimal model using the largest value.
## The final value used for the model was k = 17.
\end{verbatim}

\begin{Shaded}
\begin{Highlighting}[]
\CommentTok{\#Plotting yields Number of Neighbors Vs Accuracy (based on repeated cross validation)}
\FunctionTok{plot}\NormalTok{(knnFit)}
\end{Highlighting}
\end{Shaded}

\includegraphics{Unit_8_9_Case_Study_files/figure-latex/unnamed-chunk-10-1.pdf}

\begin{Shaded}
\begin{Highlighting}[]
\NormalTok{knnPredict }\OtherTok{\textless{}{-}} \FunctionTok{predict}\NormalTok{(knnFit,}\AttributeTok{newdata =}\NormalTok{ testing )}
\CommentTok{\#Get the confusion matrix to see accuracy value and other parameter values}
\FunctionTok{confusionMatrix}\NormalTok{(knnPredict, }\FunctionTok{factor}\NormalTok{(testing}\SpecialCharTok{$}\NormalTok{type))}
\end{Highlighting}
\end{Shaded}

\begin{verbatim}
## Confusion Matrix and Statistics
## 
##           Reference
## Prediction Ale Ipa
##        Ale  50  15
##        Ipa  13 125
##                                           
##                Accuracy : 0.8621          
##                  95% CI : (0.8069, 0.9063)
##     No Information Rate : 0.6897          
##     P-Value [Acc > NIR] : 9.876e-09       
##                                           
##                   Kappa : 0.6806          
##                                           
##  Mcnemar's Test P-Value : 0.8501          
##                                           
##             Sensitivity : 0.7937          
##             Specificity : 0.8929          
##          Pos Pred Value : 0.7692          
##          Neg Pred Value : 0.9058          
##              Prevalence : 0.3103          
##          Detection Rate : 0.2463          
##    Detection Prevalence : 0.3202          
##       Balanced Accuracy : 0.8433          
##                                           
##        'Positive' Class : Ale             
## 
\end{verbatim}

\begin{Shaded}
\begin{Highlighting}[]
\FunctionTok{mean}\NormalTok{(knnPredict }\SpecialCharTok{==}\NormalTok{ testing}\SpecialCharTok{$}\NormalTok{type)}
\end{Highlighting}
\end{Shaded}

\begin{verbatim}
## [1] 0.862069
\end{verbatim}

ABV and IBU can be used to predict if the beer is of IPA or ALE type
with an accuracy of 0.862069 which translates to
\texttt{rmean(knnPredict\ ==\ testing\$type)*100}\% accuracy rate.

\begin{enumerate}
\def\labelenumi{\arabic{enumi}.}
\setcounter{enumi}{8}
\tightlist
\item
  Knock their socks off! Find one other useful inference from the data
  that you feel Budweiser may be able to find value in. You must
  convince them why it is important and back up your conviction with
  appropriate statistical evidence.
\end{enumerate}

\begin{Shaded}
\begin{Highlighting}[]
\CommentTok{\#diffrence in meant}
\FunctionTok{t.test}\NormalTok{( ABV}\SpecialCharTok{\textasciitilde{}}\NormalTok{type, }\AttributeTok{data =}\NormalTok{ KNNDF)}
\end{Highlighting}
\end{Shaded}

\begin{verbatim}
## 
##  Welch Two Sample t-test
## 
## data:  ABV by type
## t = -17.318, df = 688.65, p-value < 2.2e-16
## alternative hypothesis: true difference in means between group Ale and group Ipa is not equal to 0
## 95 percent confidence interval:
##  -0.01440100 -0.01146818
## sample estimates:
## mean in group Ale mean in group Ipa 
##        0.05585827        0.06879286
\end{verbatim}

There is a claim that there is no difference in ABV mean between IPA and
Ale beer type. The t-test shows a p-value less than 0.05 significance
level. There we reject the null hypothesis and conclude that indeed
there is a difference in ABV mean of IPA and Ale beer type.

\hypertarget{conclusion.}{%
\subsection{Conclusion.}\label{conclusion.}}

Colorado is the leading state in number of breweries. ABV and IBU have a
positive correlation. There are both important variables in predicting
the type of beer. Finally, some beers have higher ABV than others.

\end{document}
